	\abstract{According to the United Nations Office for Disaster Risk Reduction\cite{cred2015human}, the number of global weather-related disasters in the past decades has been estimated at 90\% of the total number of disasters. These are likely to increase in severity and frequency with current and forecasted global climate changes. Disaster risk monitoring using space technology and machine learning has become essential for minimizing and managing the consequences of natural disasters. However, flood assessment and data collection remain deficient in many parts of developing countries. Recent studies in Cameroon have identified socio-economic limitations and low adaptive capacity to manage floods that threaten and expose populations to vulnerability and danger. The increasing availability of smartphones and social media data allows individuals to directly document floods in real-time and otherwise poorly observed areas. However, these data are rarely used for flood assessment purposes in developing countries. This information, in synergy with remote sensing, can help disaster managers and rescuers determine routes and maps to support flood response as well as post-flood activities, such as calibrating flood hydrodynamic models. The paper will focus on integrating community social media with satellite remote sensing data to assess and assist in flood disaster emergency response and preparedness in Douala Estuary in Cameroon. \\
   \textbf{Keywords:} social media, disaster management, remote sensing, Cameroon
}