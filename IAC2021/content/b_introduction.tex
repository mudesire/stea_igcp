\section{Introduction}
Increasing Climate change will cause frequent natural disasters, human and material losses worldwide, and will severely affect developing countries. The number of global weather disasters during the period 1995-2015 is estimated at 90\% of the total number of disasters\cite{cred2015human}. During this period, floods affected 2.3 billion people worldwide, with more people affected in Asia and Africa than in other continents\cite{wahlstrom_human_2015}. 

Recent studies in northern Cameroon show that environmental factors, socio-economic constraints and lack of adaptability to flood management endanger and expose the population to vulnerabilities and risks\cite{bang_irony_2017,bang_evaluating_2019}. The international disaster database EM-DAT reports 16 floods in Cameroon between 1988 and 2017, which killed 131 people and affected nearly 400,000\cite{shen2019spatial}. The limited risk assessment is mainly due to the lack of institutions for flood prevention, immediate disaster protection and mitigation, and the financial capacity of highly trained personnel. However, the addition of data sources and big data can enable real-time flood assessment while significantly reducing costs\cite{ towe2020rethinking}. 

As demonstrated\cite{notti_potential_2018,panchal_flooding_2019}, open access satellite images widely used for flood monitoring may have some limitations (spatio-temporal resolution, visit time, etc.).  However, they can be complemented by auxiliary data. Communication via smartphones, social networks are emerging as reliable and inexpensive auxiliary data used by the scientific community for disaster assessment, especially floods \cite{le2016crowdsourced}. In emergencies, social media platforms such as Twitter have become a tool for exchanging information at the community level \cite{lacassin2020rapid}. Twitter can be used to improve the effectiveness of social reaction, awareness, and relief efforts when combined with existing crisis management methodologies, machine learning algorithms \cite{singh2019event} and supported community-level training \cite{carley2016crowd}. When satellite imagery is not available or if images are obscured by clouds, Twitter data can also provide real-time damage assessments of disaster situations providing advantages over other existing methods \cite{david2016tweeting, baranowski2020social}. 

This paper focuses on the use of Sentinel-1 Synthetic Aperture Radar (SAR) data in Douala, Cameroon to take advantage of its active illumination to obtain day/night and all-weather data and also seeks to use new auxiliary data based on social media which are very abundant on the continent, but not valued. 

The rest of the paper is organized as follows. Section 2 describes the study area while highlighting the anthropological causes of flooding in Douala. Section 3 presents the methods used to process radar images and social media data. Section 4 presents the results and Section 5 concludes with perspectives on actions to mitigate flooding in the region.