\section{Conclusion}
With climate change, floods natural disasters caused by storms and torrential rains are increasingly common and affect almost all populations and regions of the planet. In developing countries, managing these large-scale events is amplifying the socio-economic difficulties that these countries face. Flood maps are powerful disaster response planning tool for the immediate concern of human life, settlements and infrastructures. In this work, we used an open access tool which help in assessing the history of floods and the affected areas. In addition, we discuss the values, challenges, and possible advances of social networks use to leverage flood response strategies.

Twitter usage in Cameroon is still fairly limited compared to other developed countries. However, the data still allows us to learn about flood events that are not formally documented and associated sentiments from related individuals. Further analysis to compare twitter usage during flood events with regular scenarios would provide us with a better understand of how twitter and other social media can be adapted to assist with disaster managements, from early warning and mitigation to response and recovery.


